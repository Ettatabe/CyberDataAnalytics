\section*{Introduction}

To be able to safely and trustingly perform online payments, it is important to be able to identify fraudulent transactions. But identifying fraudulent transactions in a sea of valid ones is a difficult task, mainly due to two reasons. First, all transactions look alike with little to distinguish fraudulent transactions from valid ones. And second, the number of fraudulent data points is extremely small in comparison to the entire set of transaction data. This asymmetry in class probabilities makes it diffifult to create robust and reliable classifiers, leading to high false negative rates. In this report we will explore data sampling and manipulation methods to improve the data imbalance and create classifiers building upon this. For this, we will use real transaction data with anonymised data from a bank in Mexico.

For this project we used KNIME, R and several python libraries. To be able to reproduce our results, the following need to be installed.

\begin{itemize}
	\setlength\itemsep{0.05em}
	\item KNIME
	\item R
	\item Python3
	\item Scikit learn Python library
	\item NumPy Python library
	\item Pandas Python library
	\item Seaborn Python library
	\item SciPy Python library
\end{itemize}